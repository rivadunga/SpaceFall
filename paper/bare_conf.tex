
%% bare_conf.tex
%% V1.4b
%% 2015/08/26
%% by Michael Shell
%% See:
%% http://www.michaelshell.org/
%% for current contact information.
%%
%% This is a skeleton file demonstrating the use of IEEEtran.cls
%% (requires IEEEtran.cls version 1.8b or later) with an IEEE
%% conference paper.
%%
%% Support sites:
%% http://www.michaelshell.org/tex/ieeetran/
%% http://www.ctan.org/pkg/ieeetran
%% and
%% http://www.ieee.org/

%%*************************************************************************
%% Legal Notice:
%% This code is offered as-is without any warranty either expressed or
%% implied; without even the implied warranty of MERCHANTABILITY or
%% FITNESS FOR A PARTICULAR PURPOSE! 
%% User assumes all risk.
%% In no event shall the IEEE or any contributor to this code be liable for
%% any damages or losses, including, but not limited to, incidental,
%% consequential, or any other damages, resulting from the use or misuse
%% of any information contained here.
%%
%% All comments are the opinions of their respective authors and are not
%% necessarily endorsed by the IEEE.
%%
%% This work is distributed under the LaTeX Project Public License (LPPL)
%% ( http://www.latex-project.org/ ) version 1.3, and may be freely used,
%% distributed and modified. A copy of the LPPL, version 1.3, is included
%% in the base LaTeX documentation of all distributions of LaTeX released
%% 2003/12/01 or later.
%% Retain all contribution notices and credits.
%% ** Modified files should be clearly indicated as such, including  **
%% ** renaming them and changing author support contact information. **
%%*************************************************************************


% *** Authors should verify (and, if needed, correct) their LaTeX system  ***
% *** with the testflow diagnostic prior to trusting their LaTeX platform ***
% *** with production work. The IEEE's font choices and paper sizes can   ***
% *** trigger bugs that do not appear when using other class files.       ***                          ***
% The testflow support page is at:
% http://www.michaelshell.org/tex/testflow/



\documentclass[12pt, conference]{IEEEtran}

\usepackage{graphicx}
\graphicspath{ {images/} }

% Some Computer Society conferences also require the compsoc mode option,
% but others use the standard conference format.
%
% If IEEEtran.cls has not been installed into the LaTeX system files,
% manually specify the path to it like:
% \documentclass[conference]{../sty/IEEEtran}





% Some very useful LaTeX packages include:
% (uncomment the ones you want to load)


% *** MISC UTILITY PACKAGES ***
%
%\usepackage{ifpdf}
% Heiko Oberdiek's ifpdf.sty is very useful if you need conditional
% compilation based on whether the output is pdf or dvi.
% usage:
% \ifpdf
%   % pdf code
% \else
%   % dvi code
% \fi
% The latest version of ifpdf.sty can be obtained from:
% http://www.ctan.org/pkg/ifpdf
% Also, note that IEEEtran.cls V1.7 and later provides a builtin
% \ifCLASSINFOpdf conditional that works the same way.
% When switching from latex to pdflatex and vice-versa, the compiler may
% have to be run twice to clear warning/error messages.






% *** CITATION PACKAGES ***
%
%\usepackage{cite}
% cite.sty was written by Donald Arseneau
% V1.6 and later of IEEEtran pre-defines the format of the cite.sty package
% \cite{} output to follow that of the IEEE. Loading the cite package will
% result in citation numbers being automatically sorted and properly
% "compressed/ranged". e.g., [1], [9], [2], [7], [5], [6] without using
% cite.sty will become [1], [2], [5]--[7], [9] using cite.sty. cite.sty's
% \cite will automatically add leading space, if needed. Use cite.sty's
% noadjust option (cite.sty V3.8 and later) if you want to turn this off
% such as if a citation ever needs to be enclosed in parenthesis.
% cite.sty is already installed on most LaTeX systems. Be sure and use
% version 5.0 (2009-03-20) and later if using hyperref.sty.
% The latest version can be obtained at:
% http://www.ctan.org/pkg/cite
% The documentation is contained in the cite.sty file itself.






% *** GRAPHICS RELATED PACKAGES ***
%
\ifCLASSINFOpdf
  % \usepackage[pdftex]{graphicx}
  % declare the path(s) where your graphic files are
  % \graphicspath{{../pdf/}{../jpeg/}}
  % and their extensions so you won't have to specify these with
  % every instance of \includegraphics
  % \DeclareGraphicsExtensions{.pdf,.jpeg,.png}
\else
  % or other class option (dvipsone, dvipdf, if not using dvips). graphicx
  % will default to the driver specified in the system graphics.cfg if no
  % driver is specified.
  % \usepackage[dvips]{graphicx}
  % declare the path(s) where your graphic files are
  % \graphicspath{{../eps/}}
  % and their extensions so you won't have to specify these with
  % every instance of \includegraphics
  % \DeclareGraphicsExtensions{.eps}
\fi
% graphicx was written by David Carlisle and Sebastian Rahtz. It is
% required if you want graphics, photos, etc. graphicx.sty is already
% installed on most LaTeX systems. The latest version and documentation
% can be obtained at: 
% http://www.ctan.org/pkg/graphicx
% Another good source of documentation is "Using Imported Graphics in
% LaTeX2e" by Keith Reckdahl which can be found at:
% http://www.ctan.org/pkg/epslatex
%
% latex, and pdflatex in dvi mode, support graphics in encapsulated
% postscript (.eps) format. pdflatex in pdf mode supports graphics
% in .pdf, .jpeg, .png and .mps (metapost) formats. Users should ensure
% that all non-photo figures use a vector format (.eps, .pdf, .mps) and
% not a bitmapped formats (.jpeg, .png). The IEEE frowns on bitmapped formats
% which can result in "jaggedy"/blurry rendering of lines and letters as
% well as large increases in file sizes.
%
% You can find documentation about the pdfTeX application at:
% http://www.tug.org/applications/pdftex





% *** MATH PACKAGES ***
%
%\usepackage{amsmath}
% A popular package from the American Mathematical Society that provides
% many useful and powerful commands for dealing with mathematics.
%
% Note that the amsmath package sets \interdisplaylinepenalty to 10000
% thus preventing page breaks from occurring within multiline equations. Use:
%\interdisplaylinepenalty=2500
% after loading amsmath to restore such page breaks as IEEEtran.cls normally
% does. amsmath.sty is already installed on most LaTeX systems. The latest
% version and documentation can be obtained at:
% http://www.ctan.org/pkg/amsmath





% *** SPECIALIZED LIST PACKAGES ***
%
%\usepackage{algorithmic}
% algorithmic.sty was written by Peter Williams and Rogerio Brito.
% This package provides an algorithmic environment fo describing algorithms.
% You can use the algorithmic environment in-text or within a figure
% environment to provide for a floating algorithm. Do NOT use the algorithm
% floating environment provided by algorithm.sty (by the same authors) or
% algorithm2e.sty (by Christophe Fiorio) as the IEEE does not use dedicated
% algorithm float types and packages that provide these will not provide
% correct IEEE style captions. The latest version and documentation of
% algorithmic.sty can be obtained at:
% http://www.ctan.org/pkg/algorithms
% Also of interest may be the (relatively newer and more customizable)
% algorithmicx.sty package by Szasz Janos:
% http://www.ctan.org/pkg/algorithmicx




% *** ALIGNMENT PACKAGES ***
%
%\usepackage{array}
% Frank Mittelbach's and David Carlisle's array.sty patches and improves
% the standard LaTeX2e array and tabular environments to provide better
% appearance and additional user controls. As the default LaTeX2e table
% generation code is lacking to the point of almost being broken with
% respect to the quality of the end results, all users are strongly
% advised to use an enhanced (at the very least that provided by array.sty)
% set of table tools. array.sty is already installed on most systems. The
% latest version and documentation can be obtained at:
% http://www.ctan.org/pkg/array


% IEEEtran contains the IEEEeqnarray family of commands that can be used to
% generate multiline equations as well as matrices, tables, etc., of high
% quality.




% *** SUBFIGURE PACKAGES ***
%\ifCLASSOPTIONcompsoc
%  \usepackage[caption=false,font=normalsize,labelfont=sf,textfont=sf]{subfig}
%\else
%  \usepackage[caption=false,font=footnotesize]{subfig}
%\fi
% subfig.sty, written by Steven Douglas Cochran, is the modern replacement
% for subfigure.sty, the latter of which is no longer maintained and is
% incompatible with some LaTeX packages including fixltx2e. However,
% subfig.sty requires and automatically loads Axel Sommerfeldt's caption.sty
% which will override IEEEtran.cls' handling of captions and this will result
% in non-IEEE style figure/table captions. To prevent this problem, be sure
% and invoke subfig.sty's "caption=false" package option (available since
% subfig.sty version 1.3, 2005/06/28) as this is will preserve IEEEtran.cls
% handling of captions.
% Note that the Computer Society format requires a larger sans serif font
% than the serif footnote size font used in traditional IEEE formatting
% and thus the need to invoke different subfig.sty package options depending
% on whether compsoc mode has been enabled.
%
% The latest version and documentation of subfig.sty can be obtained at:
% http://www.ctan.org/pkg/subfig




% *** FLOAT PACKAGES ***
%
%\usepackage{fixltx2e}
% fixltx2e, the successor to the earlier fix2col.sty, was written by
% Frank Mittelbach and David Carlisle. This package corrects a few problems
% in the LaTeX2e kernel, the most notable of which is that in current
% LaTeX2e releases, the ordering of single and double column floats is not
% guaranteed to be preserved. Thus, an unpatched LaTeX2e can allow a
% single column figure to be placed prior to an earlier double column
% figure.
% Be aware that LaTeX2e kernels dated 2015 and later have fixltx2e.sty's
% corrections already built into the system in which case a warning will
% be issued if an attempt is made to load fixltx2e.sty as it is no longer
% needed.
% The latest version and documentation can be found at:
% http://www.ctan.org/pkg/fixltx2e


%\usepackage{stfloats}
% stfloats.sty was written by Sigitas Tolusis. This package gives LaTeX2e
% the ability to do double column floats at the bottom of the page as well
% as the top. (e.g., "\begin{figure*}[!b]" is not normally possible in
% LaTeX2e). It also provides a command:
%\fnbelowfloat
% to enable the placement of footnotes below bottom floats (the standard
% LaTeX2e kernel puts them above bottom floats). This is an invasive package
% which rewrites many portions of the LaTeX2e float routines. It may not work
% with other packages that modify the LaTeX2e float routines. The latest
% version and documentation can be obtained at:
% http://www.ctan.org/pkg/stfloats
% Do not use the stfloats baselinefloat ability as the IEEE does not allow
% \baselineskip to stretch. Authors submitting work to the IEEE should note
% that the IEEE rarely uses double column equations and that authors should try
% to avoid such use. Do not be tempted to use the cuted.sty or midfloat.sty
% packages (also by Sigitas Tolusis) as the IEEE does not format its papers in
% such ways.
% Do not attempt to use stfloats with fixltx2e as they are incompatible.
% Instead, use Morten Hogholm'a dblfloatfix which combines the features
% of both fixltx2e and stfloats:
%
% \usepackage{dblfloatfix}
% The latest version can be found at:
% http://www.ctan.org/pkg/dblfloatfix




% *** PDF, URL AND HYPERLINK PACKAGES ***
%
%\usepackage{url}
% url.sty was written by Donald Arseneau. It provides better support for
% handling and breaking URLs. url.sty is already installed on most LaTeX
% systems. The latest version and documentation can be obtained at:
% http://www.ctan.org/pkg/url
% Basically, \url{my_url_here}.




% *** Do not adjust lengths that control margins, column widths, etc. ***
% *** Do not use packages that alter fonts (such as pslatex).         ***
% There should be no need to do such things with IEEEtran.cls V1.6 and later.
% (Unless specifically asked to do so by the journal or conference you plan
% to submit to, of course. )


% correct bad hyphenation here
\hyphenation{op-tical net-works semi-conduc-tor}


\begin{document}
%
% paper title
% Titles are generally capitalized except for words such as a, an, and, as,
% at, but, by, for, in, nor, of, on, or, the, to and up, which are usually
% not capitalized unless they are the first or last word of the title.
% Linebreaks \\ can be used within to get better formatting as desired.
% Do not put math or special symbols in the title.
\title{Space Fall Game Development}


% author names and affiliations
% use a multiple column layout for up to three different
% affiliations
\author{\IEEEauthorblockN{Erick Rivas}
\IEEEauthorblockA{School of Engineering\\
ITESM CSF\\
M\'{e}xico, D.F.\\
Email: erick.cecyt@gmail.com}
\and
\IEEEauthorblockN{Alejandra Espidio}
\IEEEauthorblockA{School of Engineering\\
ITESM CSF\\
M\'{e}xico, D.F.\\
Email: aleespidio@gmail.com}
\and
\IEEEauthorblockN{Jonathan Ginsburg}
\IEEEauthorblockA{School of Engineering\\
ITESM CSF\\
M\'{e}xico, D.F.\\
Email: jginsburgn@gmail.com}}

% conference papers do not typically use \thanks and this command
% is locked out in conference mode. If really needed, such as for
% the acknowledgment of grants, issue a \IEEEoverridecommandlockouts
% after \documentclass

% for over three affiliations, or if they all won't fit within the width
% of the page, use this alternative format:
% 
%\author{\IEEEauthorblockN{Michael Shell\IEEEauthorrefmark{1},
%Homer Simpson\IEEEauthorrefmark{2},
%James Kirk\IEEEauthorrefmark{3}, 
%Montgomery Scott\IEEEauthorrefmark{3} and
%Eldon Tyrell\IEEEauthorrefmark{4}}
%\IEEEauthorblockA{\IEEEauthorrefmark{1}School of Electrical and Computer Engineering\\
%Georgia Institute of Technology,
%Atlanta, Georgia 30332--0250\\ Email: see http://www.michaelshell.org/contact.html}
%\IEEEauthorblockA{\IEEEauthorrefmark{2}Twentieth Century Fox, Springfield, USA\\
%Email: homer@thesimpsons.com}
%\IEEEauthorblockA{\IEEEauthorrefmark{3}Starfleet Academy, San Francisco, California 96678-2391\\
%Telephone: (800) 555--1212, Fax: (888) 555--1212}
%\IEEEauthorblockA{\IEEEauthorrefmark{4}Tyrell Inc., 123 Replicant Street, Los Angeles, California 90210--4321}}




% use for special paper notices
%\IEEEspecialpapernotice{(Invited Paper)}




% make the title area
\maketitle

% As a general rule, do not put math, special symbols or citations
% in the abstract
\begin{abstract}

This document intends to put forward an explanation for the creation of the game called Space Fall. It is a detailed report and is accompanied by images and C\# code. The game uses a new technique of path selection for characters in a three-dimensional environment; it can be categorised as both, an RPG and an RTS, among others. The developing game engine was Unity 5.\\ \\
Keywords: path selection, ai, unity, rpg, rts, game development.\\
Palabras Clave: selecci\'{o}n de caminos, ia, unity, desarrollo de videojuegos.
\end{abstract}

% no keywords




% For peer review papers, you can put extra information on the cover
% page as needed:
% \ifCLASSOPTIONpeerreview
% \begin{center} \bfseries EDICS Category: 3-BBND \end{center}
% \fi
%
% For peerreview papers, this IEEEtran command inserts a page break and
% creates the second title. It will be ignored for other modes.
\IEEEpeerreviewmaketitle



\section{Introducci\'{o}n}

Space Fall es un videojuego para ordenadores de simulaci\'{o}n. La primer versi\'{o}n de este juego est\'{a} para el computador aplicando las teclas "WASD" para su movilidad. Del g\'{e}nero RPG, RTS, Action, Adventure, Fighting, Puzzle, Shooter, Sports y Simulaci\'{o}n arcade, el videojuego est\'{a} desarrollado en una \'{e}poca futurista donde la misi\'{o}n es recolectar informaci\'{o}n de diversos planetas para que la especie humana prevalezca. El \'{u}nico objetivo del juego es el de entretener a los usuarios de cualquier g\'{e}nero y edad, y que disfruten combatiendo obst\'{a}culos para ser h\'{e}roes. Cada misi\'{o}n consiste en una expedici\'{o}n a\'{e}rea muy similar a un salto en ca\'{i}da libre. El personaje principal es un humanoide, manipulado por el usuario, ser\'{a} el encargado de cumplir las tres diferentes misiones, una misi\'{o}n por cada planeta. Para crear el videojuego se us\'{o} el engine Unity versi\'{o}n 5.0 y se tomaron en cuenta diferentes aspectos como: el dise\~{n}o gr\'{a}fico de la interfaz, personaje y objetos, y una serie de algoritmos utilizados para su desarrollo.\\

\hfill E.R., A.E., y J.G.
 
\hfill Diciembre 3, 2015

\section{Procedimiento}

\subsection{Men\'{u}}

El men\'{u} se realiz\'{o} utilizando objetos de la librer\'{i}a de interfaz de usuario UI de Unity. Algunos de estos objetos fueron botones, p\'{a}neles y textos. Adicionalmente se sigui\'{o} la gu\'{i}a referida en bibliograf\'{i}a para la creaci\'{o}n de men\'{u}s. Por otra parte el Asset Collection de Galactic Objects fue utilizado para toda la parte visual; \'{e}ste se puede descargar de la Asset Store de Unity.

B\'{a}sicamente se colocaron los objetos aludidos anteriormente en la escena de tal manera que la c\'{a}mara pudiera observarlos como es convencional para los men\'{u}es de videojuegos. Posteriormente, los botones, se conectaron a una funci\'{o}n que tiene como argumento un n\'{u}mero entero el cual hace referencia al \'{i}ndice de escena de las opciones del videojuego de unity (Build Settings). Dicha funci\'{o}n hace el cambio de escena correspondiente al nivel seleccionado o a la pantalla de puntajes altos.

\subsection{Algoritmo de Selecci\'{o}n de Caminos}

El algoritmo de selecci\'{o}n de caminos fue dise\~{n}ado mediante t\'{e}cnicas de dise\~{n}o que se separan modularmente. Estos fueron: algoritmos \'{a}vidos y backtracking. Algoritmos \'{a}vidos se refiere al paradigma de dise\~{n}o que propone tomar la soluci\'{o}n \'{o}ptima local. Backtracking es el nombre que se le da a las implementaciones recursivas en grafos por profundidad.

De manera breve, el algoritmo establece, en un espacio tridimensional, una esfera con centro en el personaje principal. Dicha esfera es cortada por planos en varias direcciones con v\'{e}rtice de uni\'{o}n en el mismo centro que la misma. Esto produce un n\'{u}mero de intersecciones, las cuales se toman como nodos de un grafo y las aristas se establecen de manera que los nodos no queden desconectados, sin llegar a ser una malla.

Respecto a los algoritmos utilizados en el videojuegos, se utilizar\'{a} una adaptaci\'{o}n del algoritmo Dijkstra con el objetivo de brindar mayor realismo a las trayectorias seguidas por los enemigos. De igual manera, se emplear\'{a} el Algoritmo de interpolaci\'{o}n de Newton para ejecutar aproximaciones que permitan ponderar la dificultad del juego.

La metodolog\'{i}a que utilizaremos durante el desarrollo del videojuego ser\'{a} Scrum. Para adaptar esta metodolog\'{i}a al videojuegos, dise\~{n}aremos cada Sprint para que se adecuen a las etapa del juego, ya sea l\'{o}gica (Programaci\'{o}n y testing), gr\'{a}fica (Modelado, animaci\'{o}n, etc) o integradora. Asimismo, asignaremos los roles del equipo de acuerdo a las caracter\'{i}sticas de la iteraci\'{o}n , por ejemplo, si estamos en un Sprint que requiera muchas cuestiones gr\'{a}ficas el ScrumMaster ser\'{a} el dise\~{n}ador (Alejandra) y los programadores formar\'{i}an parte del equipo de desarrollo.

\subsection{Dise\~{n}o Gr\'{a}fico}

El personaje principal que se usa en el videojuego, es un humanoide en 3D para crear animaciones cercanas a la realidad de la ca\'{i}da libre; cuando corre desde la nave espacial de Neways, cuando est\'{a} cayendo por el espacio y cuando esquiva hacia la izquierda y derecha. Con el fin de que el usuario se sienta atra\'{i}do y a gusto con el dise\~{n}o del personaje, se escogi\'{o} un humanoide que fuera amigable y de la mano con la historia del juego. El nombre del humanoide que se obtuvo de la p\'{a}gina web Creative Crash es roboter 2.0, animado en el programa de Autodesk, Maya 2015. 

\begin{figure}[h]
\caption{Humanoide}
\centering
\includegraphics[width=0.5\textwidth]{robotfront}
\end{figure}

Asimismo los dem\'{a}s objetos est\'{a}n creados con el mismo programa, sus colores y formas fueron escogidos as\'{i} para que tuvieran armon\'{i}a con el personaje principal de la historia y la historia. 
La nave espacial est\'{a} basada en los dise\~{n}os de las naves futuristas que aparecen en pel\'{i}culas como Star Trek o Star Wars.

\begin{figure}[h]
\caption{Nave Espacial}
\centering
\includegraphics[width=0.5\textwidth]{ship}
\end{figure}

Para que el personaje obtenga la energ\'{i}a, necesita capturar las latas de aceite que el programa Neways le pone en su camino y poder terminar la misi\'{o}n

\begin{figure}[h]
\caption{Lata de Aceitel}
\centering
\includegraphics[width=0.5\textwidth]{can}
\end{figure}

Cuando el humanoide termine la misi\'{o}n aterrizar\'{a} en una plataforma de Neways

\begin{figure}[h]
\caption{Plataforma de Aterrizaje}
\centering
\includegraphics[width=0.5\textwidth]{platform}
\end{figure}

Los logotipos est\'{a}n creados con el programa de Adobe Illustrator. Neways es creado por la NASA para cumplir la misi\'{o}n de que la especie humana prevalezca, por eso su, logo simula un planeta y contiene un aro, haciendo que el logo se parezca al de la NASA.

\begin{figure}[h]
\caption{Logo de Neways}
\centering
\includegraphics[width=0.5\textwidth]{neways}
\end{figure}

El logotipo del videojuego fue creado de acuerdo con la idea futurista y espacial. La tipograf\'{i}a que se usa, SPACE AGE, y el color verde hacen \'{e}nfasis a la idea adem\'{a}s de que crea armon\'{i}a con la interfaz

\begin{figure}[h]
\caption{Logo de Spacefall}
\centering
\includegraphics[width=0.5\textwidth]{spacefall}
\end{figure}

Se usa de los assets de Unity el skybox en 2D del espacio y un mundo en 3D del que el humanoide tiene que recolectar la informaci\'{o}n pero mientras lo hace aparecen asteroides que tiene que esquivar y hay unos aros azules que cambian de color a verde cuando el humanoide entra en ellos para obtener puntos

Para administrar y almacenar la vida del personaje se crea un control. Su dise\~{n}o est\'{a} basado en la cara del humanoide, su ojo es donde se almacena la vida poni\'{e}ndolo de color azul o blanco cuando est\'{a} vac\'{i}a, todo para generar armon\'{i}a en la interfaz

\begin{figure}[h]
\caption{Vida Llena}
\centering
\includegraphics[width=0.5\textwidth]{fulllife}
\end{figure}

\begin{figure}[h]
\caption{Vida Vac\'{i}a}
\centering
\includegraphics[width=0.5\textwidth]{emptylife}
\end{figure}

\section{Resultados}

El videojuego result\'{o} funcionar de la mejor manera con los algoritmos mencionados y se logr\'{o} combinar con la parte gr\'{a}fica; personaje, objetos y la escena que la Asset Store de Unity nos ofrece, adem\'{a}s el audio. Como otra versi\'{o}n, se implement\'{o} el uso del Oculus. El usuario ahora podr\'{a} tener m\'{a}s contacto con el videojuego y podr\'{a} disfrutarlo mejor. Todo esto con el fin de generar un juego que es f\'{a}cil de entender, usar y que cualquiera puede utilizar, al mismo tiempo genera una armon\'{i}a visual. 

\section{Conclusi\'{o}n}

La realizaci\'{o}n del presente videojuego fue altamente multidisciplinaria. Como equipo central, se unieron las habilidades de una aspirante a la Licenciatura de Artes Digitales, Alejandra Espidio, y las de dos aspirantes a la Ingenier\'{i}a en Tecnolog\'{i}as Computacionales, Erick Rivas y Jonathan Ginsburg. Tambi\'{e}n se utilizaron los conocimientos de terceros a los que se alude en los agradecimientos. Por tanto, Space Fall es un juego de arte e ingenier\'{i}a en armon\'{i}a. Tiene implementaciones de algoritmos innovadores y modelos con mucho toque art\'{i}stico.

Adicionalmente, el algoritmo de selecci\'{o}n de caminos en un espacio tridimensional abierto es de autor\'{i}a propia y vuelve a plantear las posibilidades de hacer dicha tarea. B\'{a}sicamente consiste en considerar un grafo, del cual los nodos se obtienen por cruces de planos en una esfera con centro en el punto pivote del personaje principal, y las aristas son seleccionadas de manera que dos nodos no queden separados sin necesidad de hacer una maya. Por lo tanto, se tiene un punto apuntando a casi todas las direcciones discretas en las que se puede mover el personaje. Esto genera que los puntos puedan tomar valores correspondientes a lo cerca o lejos que se encuentran de objetivos que se quieran alcanzar y que se quieran evadir. De esta manera el personaje tiene una tendencia a ir por los nodos que tengan peso favorable.

Se espera continuar con el desarrollo del algoritmo central de nuestro videojuego para el futuro. As\'{i} tambi\'{e}n, consideramos la posibilidad de publicar el videojuego en alguna plataforma para poder conocer el pensamiento de los jugadores.

\section{Bibliograf\'{i}a}

\begingroup

    \fontsize{10pt}{12pt}\selectfont
    Unity Tutorials. \emph{Creating a Scene Selection Menu.} https://unity3d.com/learn/tutorials/modules/beginner/live-training-archive/creating-a-scene-menu. Julio 24 de 2015. Web.
    \\ \\
    NASA. Kepler-452b: Earth's Bigger, Older Cousin -- Briefing Materials. https://www.nasa.gov/keplerbriefing0723.
    \\ \\
    Wikipedia. Gleise 667 Cc. https://en.wikipedia.org/wiki/Gliese\_667\_Cc.
    \\ \\
    The Guardian. Kepler 438b: Most Earth-like planet ever discovered could be home for alien life. http://www.theguardian.com/science/2015/jan/06/earth-like-planet-alien-life-kepler-438b.
    \\ \\
    http://www.creativecrash.com/maya/downloads/character-rigs/c/roboter

\endgroup

\section*{Agradecimientos}

Para la realizaci\'{o}n del presente videojuego fue crucial contar con la ayuda y el entusiasmo de varias personas. La experiencia con las diversas herramientas que fueron utilizadas por parte de los que ayudaron fue imprescindible. Por otra parte, su amor por la materia correspondiente nos llev\'{o} a aprender de ellos. Mencionamos a algunas de estas personas: Samuel Ginsburg, Moises Alencastre y, el estudiante de doctorado, Octavio.

\section{Anexos}

C\'{o}digo de selecci\'{o}n de caminos en C\#: 

\includegraphics[width=0.5\textwidth]{firstsnippet}
\includegraphics[width=0.5\textwidth]{secondsnippet}
\includegraphics[width=0.5\textwidth]{thirdsnippet}

% An example of a floating figure using the graphicx package.
% Note that \label must occur AFTER (or within) \caption.
% For figures, \caption should occur after the \includegraphics.
% Note that IEEEtran v1.7 and later has special internal code that
% is designed to preserve the operation of \label within \caption
% even when the captionsoff option is in effect. However, because
% of issues like this, it may be the safest practice to put all your
% \label just after \caption rather than within \caption{}.
%
% Reminder: the "draftcls" or "draftclsnofoot", not "draft", class
% option should be used if it is desired that the figures are to be
% displayed while in draft mode.
%
%\begin{figure}[!t]
%\centering
%\includegraphics[width=2.5in]{myfigure}
% where an .eps filename suffix will be assumed under latex, 
% and a .pdf suffix will be assumed for pdflatex; or what has been declared
% via \DeclareGraphicsExtensions.
%\caption{Simulation results for the network.}
%\label{fig_sim}
%\end{figure}

% Note that the IEEE typically puts floats only at the top, even when this
% results in a large percentage of a column being occupied by floats.


% An example of a double column floating figure using two subfigures.
% (The subfig.sty package must be loaded for this to work.)
% The subfigure \label commands are set within each subfloat command,
% and the \label for the overall figure must come after \caption.
% \hfil is used as a separator to get equal spacing.
% Watch out that the combined width of all the subfigures on a 
% line do not exceed the text width or a line break will occur.
%
%\begin{figure*}[!t]
%\centering
%\subfloat[Case I]{\includegraphics[width=2.5in]{box}%
%\label{fig_first_case}}
%\hfil
%\subfloat[Case II]{\includegraphics[width=2.5in]{box}%
%\label{fig_second_case}}
%\caption{Simulation results for the network.}
%\label{fig_sim}
%\end{figure*}
%
% Note that often IEEE papers with subfigures do not employ subfigure
% captions (using the optional argument to \subfloat[]), but instead will
% reference/describe all of them (a), (b), etc., within the main caption.
% Be aware that for subfig.sty to generate the (a), (b), etc., subfigure
% labels, the optional argument to \subfloat must be present. If a
% subcaption is not desired, just leave its contents blank,
% e.g., \subfloat[].


% An example of a floating table. Note that, for IEEE style tables, the
% \caption command should come BEFORE the table and, given that table
% captions serve much like titles, are usually capitalized except for words
% such as a, an, and, as, at, but, by, for, in, nor, of, on, or, the, to
% and up, which are usually not capitalized unless they are the first or
% last word of the caption. Table text will default to \footnotesize as
% the IEEE normally uses this smaller font for tables.
% The \label must come after \caption as always.
%
%\begin{table}[!t]
%% increase table row spacing, adjust to taste
%\renewcommand{\arraystretch}{1.3}
% if using array.sty, it might be a good idea to tweak the value of
% \extrarowheight as needed to properly center the text within the cells
%\caption{An Example of a Table}
%\label{table_example}
%\centering
%% Some packages, such as MDW tools, offer better commands for making tables
%% than the plain LaTeX2e tabular which is used here.
%\begin{tabular}{|c||c|}
%\hline
%One & Two\\
%\hline
%Three & Four\\
%\hline
%\end{tabular}
%\end{table}


% Note that the IEEE does not put floats in the very first column
% - or typically anywhere on the first page for that matter. Also,
% in-text middle ("here") positioning is typically not used, but it
% is allowed and encouraged for Computer Society conferences (but
% not Computer Society journals). Most IEEE journals/conferences use
% top floats exclusively. 
% Note that, LaTeX2e, unlike IEEE journals/conferences, places
% footnotes above bottom floats. This can be corrected via the
% \fnbelowfloat command of the stfloats package.
% conference papers do not normally have an appendix


% use section* for acknowledgment

% trigger a \newpage just before the given reference
% number - used to balance the columns on the last page
% adjust value as needed - may need to be readjusted if
% the document is modified later
%\IEEEtriggeratref{8}
% The "triggered" command can be changed if desired:
%\IEEEtriggercmd{\enlargethispage{-5in}}

% references section

% can use a bibliography generated by BibTeX as a .bbl file
% BibTeX documentation can be easily obtained at:
% http://mirror.ctan.org/biblio/bibtex/contrib/doc/
% The IEEEtran BibTeX style support page is at:
% http://www.michaelshell.org/tex/ieeetran/bibtex/
%\bibliographystyle{IEEEtran}
% argument is your BibTeX string definitions and bibliography database(s)
%\bibliography{IEEEabrv,../bib/paper}
%
% <OR> manually copy in the resultant .bbl file
% set second argument of \begin to the number of references
% (used to reserve space for the reference number labels box)

% that's all folks
\end{document}


